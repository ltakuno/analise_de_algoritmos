\documentclass{article}

\usepackage[brazil]{babel}     % Defini��o-padr�o l�ngua portuguesa
\usepackage[latin1]{inputenc}  % Para traduzir acentos
\usepackage{graphicx}
\usepackage{indentfirst}
\usepackage{float}

\usepackage{clrscode}

\usepackage{amsmath,amssymb,amsthm} \usepackage{amsfonts}
\usepackage[colorlinks,backref]{hyperref}

\setlength{\topmargin}{-0.5in}
\setlength{\textheight}{9in}
\setlength{\oddsidemargin}{-.125in}
\setlength{\evensidemargin}{-.125in}
\setlength{\textwidth}{6.75in}


\newcommand{\CWEB}{\texttt{CWEB} }
\newtheorem{definicao}{Defini��o}[section]
\newtheorem{exemplo}{Exemplo}[section]
\newtheorem{teorema}{Teorema}[section]
\newtheorem{corolario}{Corol�rio}[section]

\def\red{\color[rgb]{0.7,0,0}}
\def\green{\color[rgb]{0,.8,0}}
\def\darkgreen{\color[rgb]{0.1,0.4,0.0}}
\let\dgreen\darkgreen
\def\blue{\color[rgb]{0,0,.8}}
\def\yellow{\color[rgb]{1,1,0}}
\def\black{\color[rgb]{0,0,0}}

\def\pink{\color[rgb]{1,0,1}}
\def\brown{\color[rgb]{.5,.1,.3}}
\def\lilaz{\color[rgb]{.5,0,.5}}
\def\hmmm{\color[rgb]{.3,.1,.5}}
\def\magenta{\color[rgb]{.6,.05,.05}}

\newcommand{\uc}[1]{\centerline{\underline{#1}}}
\newcommand{\pic}[1]{\fbox{picture:{#1}}}
%\renewcommand{\bf}{\mbox{}}
\newcommand{\cP}{{\cal P}} \newcommand{\cT}{{\cal T}}
\newcommand{\add}{\mbox{\rm add}} \newcommand{\pr}{\mbox{\rm Pr}}

\def\stitle#1{\slidetitle{\red #1}\vspace{-0pt}}
\everymath={\blue}
\everydisplay={\blue}

\def\itemtrig{$\vartriangleright$}
\def\itemcirc{$\circ$}
\def\itemT{\item[\itemtrig]}
\def\itemC{\item[$\circ$]}

\renewcommand{\For}{\textbf{\blue for} }
\renewcommand{\To}{\textbf{\blue to} }
\renewcommand{\Do}{\>\>\textbf{\blue do}\hspace*{-0.7em}\'\addtocounter{indent}{1}}
\renewcommand{\While}{\textbf{\blue while} }
\renewcommand{\If}{\textbf{\blue if} }
\renewcommand{\Then}{\>\textbf{\blue then}\>\addtocounter{indent}{1}}
\renewcommand{\Else}{\kill\addtocounter{indent}{-1}\liprint\>\textbf{\blue else}\>\addtocounter{indent}{1}}
\renewcommand{\ElseIf}{\kill\addtocounter{indent}{-1}\liprint\textbf{\blue elseif} }
\renewcommand{\ElseNoIf}{\kill\addtocounter{indent}{-1}\liprint\textbf{\blue else} \addtocounter{indent}{1}}
\renewcommand{\Return}{\textbf{\blue return} }



\begin{document}


\bibliographystyle{plain}       % estilo de bibliografia

\title{Centro Universit�rio Senac\\ Bacharelado em Ci�ncia da Computa��o \\ An�lise e projeto de algoritmos - Exerc�cios aula 01}
\author{Professor: Leonardo Takuno \\
\texttt{\{leonardo.takuno@gmail.com\}}}
\maketitle

\begin{enumerate}
\item Dado $f(n)$ determine a nota��o assint�tica
\begin{enumerate}
\item[(a)] $f(n) = 10^{80}$

Resposta: $\Theta(1)$

\item[(b)] $f(n) = (20n)^7$

Resposta: $\Theta(n^7)$

\item[(c)] $f(n) = (\log n)^{100}$

Resposta: $(\log n)^{100} = 100 \log n = \Theta(\log n)$

\item[(d)] $f(n) = \log_{\ln (5)} (\log n^{100})$

Resposta:
$\log_{\ln (5)} (\log n^{100}) = \frac{1}{\log \ln(5)} \log(\log n^{100}) = \frac{1}{\log \ln(5)} \log(100 \log n) = \frac{1}{\log \ln(5)} (\log(100) + \log \log n))$

$\therefore f(n) = \Theta(\log \log n)$


\item[(e)] $f(n) = \log \left(\left( \begin{array}{c} n \\ n/2 \end{array} \right)\right)$

Obs. 1:  $\left( \begin{array}{c} n \\ n/2 \end{array} \right) = \displaystyle \frac{n!}{\frac{n!}{2} \cdot \frac{n!}{2}}$


Obs. 2:

F�rmula de Stirling: $\displaystyle n! \approx \sqrt{2 \pi n} \left( \frac{n}{e} \right)^n, e = 2,718$
 
Resposta:


\begin{tabular}{ll}
$=\log \left(\left( \begin{array}{c} n \\ n/2 \end{array} \right)\right)$ & $=$ \\
$=\log \left( \frac{\sqrt{2 \pi n} \left( \frac{n}{e} \right)^n}{\left(\sqrt{2 \pi \frac{n}{2}} \left( \frac{n}{2e} \right)^{\frac{n}{2}} \right)^2} \right)$ & $=$ \\
$=\log \left( \frac{\sqrt{2 \pi n} \left( \frac{n}{e} \right)^n}{\left(\sqrt{\pi n} \left( \frac{n}{2e} \right)^{\frac{n}{2}} \right)^2} \right)$ & $=$ \\
$=\log \left( \frac{\sqrt{2 \pi n} \left( \frac{n}{e} \right)^n}{\left(\pi n \left( \frac{n}{2e} \right)^{n} \right)} \right)$ & $=$ \\
$=\log \left( \frac{\sqrt{2 \pi n} 2^n}{\left(\pi n \right)} \right)$ & $=$ \\
$=\log \left( \frac{\sqrt{2 \pi } 2^n}{\pi \sqrt{n} } \right)$ & $=$ \\
$=\log \left( \frac{ 2^n}{ \sqrt{n} } \right)$ & $=$ \\ 
$=\log 2^n - \log \sqrt{n}  $ & $=$ \\ 
$= \Theta(n) - \frac{1}{2} \log n  $ &  \\
\end{tabular}

\end{enumerate}

\item Demonstrar que 
\begin{enumerate}
\item[(a)] $n^2 + 800 = O(n^2)$ 

Prova:

$n^2 + 800 \leq n^2 + 800n^2 = 801n^2 \forall n \geq 1 \Rightarrow c = 801$ e $n_0 = 1$


Prova alternativa:

$n^2 + 800 \leq n^2 + n*n = 2 n^2 \forall n \geq 800 \Rightarrow c = 2$ e $n_0 = 800$

\item[(b)] $100n^2 = O(n^3)$

Prova:

$100n^2 \leq n*n^2 \forall n \geq 100 \Rightarrow c = 1$ e $n_0 = 100$

\item[(c)] $10n^3-3n^2 + 27 = O(n^3)$

Prova:

$10n^3-3n^2 + 27 = \leq 10n^3$ se $(3n^2 - 27) \geq 0$ ou seja

$10n^3-3n^2 + 27 = \leq 10n^3 \forall n \geq 3 \Rightarrow c = 10, n_0 = 3$

\item[(d)] $n = O(2^n)$

Prova por indu��o: 

Devemos mostrar que $n \leq 2^n \forall n \geq 1 \Rightarrow c=1$ e $n_0 = 1$.

\textbf{Base: }Se $n=1$ temos $1 \leq 2$.

\textbf{Hip�tese de indu��o:} Para $n \geq 2$ temos $(n-1) \leq 2^{n-1}$

\textbf{Passo de indu��o:} Ent�o $n \leq n + (n-2) = (n-1) + (n-1) \leq 2^{n-1} + 2^{n-1} = 2(2^{n-1}) = 2^n$.



\end{enumerate}

\end{enumerate}
\end{document}
