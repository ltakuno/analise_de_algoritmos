\documentclass{article}

\usepackage[brazil]{babel}     % Defini��o-padr�o l�ngua portuguesa
\usepackage[latin1]{inputenc}  % Para traduzir acentos
\usepackage{graphicx}
\usepackage{indentfirst}
\usepackage{float}

\usepackage{clrscode}

\usepackage{amsmath,amssymb,amsthm} \usepackage{amsfonts}
\usepackage[colorlinks,backref]{hyperref}

\setlength{\topmargin}{-0.5in}
\setlength{\textheight}{9in}
\setlength{\oddsidemargin}{-.125in}
\setlength{\evensidemargin}{-.125in}
\setlength{\textwidth}{6.75in}


\newcommand{\CWEB}{\texttt{CWEB} }
\newtheorem{definicao}{Defini��o}[section]
\newtheorem{exemplo}{Exemplo}[section]
\newtheorem{teorema}{Teorema}[section]
\newtheorem{corolario}{Corol�rio}[section]

\def\red{\color[rgb]{0.7,0,0}}
\def\green{\color[rgb]{0,.8,0}}
\def\darkgreen{\color[rgb]{0.1,0.4,0.0}}
\let\dgreen\darkgreen
\def\blue{\color[rgb]{0,0,.8}}
\def\yellow{\color[rgb]{1,1,0}}
\def\black{\color[rgb]{0,0,0}}

\def\pink{\color[rgb]{1,0,1}}
\def\brown{\color[rgb]{.5,.1,.3}}
\def\lilaz{\color[rgb]{.5,0,.5}}
\def\hmmm{\color[rgb]{.3,.1,.5}}
\def\magenta{\color[rgb]{.6,.05,.05}}

\newcommand{\uc}[1]{\centerline{\underline{#1}}}
\newcommand{\pic}[1]{\fbox{picture:{#1}}}
%\renewcommand{\bf}{\mbox{}}
\newcommand{\cP}{{\cal P}} \newcommand{\cT}{{\cal T}}
\newcommand{\add}{\mbox{\rm add}} \newcommand{\pr}{\mbox{\rm Pr}}

\def\stitle#1{\slidetitle{\red #1}\vspace{-0pt}}
\everymath={\blue}
\everydisplay={\blue}

\def\itemtrig{$\vartriangleright$}
\def\itemcirc{$\circ$}
\def\itemT{\item[\itemtrig]}
\def\itemC{\item[$\circ$]}

\renewcommand{\For}{\textbf{\blue for} }
\renewcommand{\To}{\textbf{\blue to} }
\renewcommand{\Do}{\>\>\textbf{\blue do}\hspace*{-0.7em}\'\addtocounter{indent}{1}}
\renewcommand{\While}{\textbf{\blue while} }
\renewcommand{\If}{\textbf{\blue if} }
\renewcommand{\Then}{\>\textbf{\blue then}\>\addtocounter{indent}{1}}
\renewcommand{\Else}{\kill\addtocounter{indent}{-1}\liprint\>\textbf{\blue else}\>\addtocounter{indent}{1}}
\renewcommand{\ElseIf}{\kill\addtocounter{indent}{-1}\liprint\textbf{\blue elseif} }
\renewcommand{\ElseNoIf}{\kill\addtocounter{indent}{-1}\liprint\textbf{\blue else} \addtocounter{indent}{1}}
\renewcommand{\Return}{\textbf{\blue return} }



\begin{document}


\bibliographystyle{plain}       % estilo de bibliografia

\title{Centro Universit�rio Senac\\ Bacharelado em Ci�ncia da Computa��o \\ An�lise e projeto de algoritmos}
\author{Professor: Leonardo Takuno \\
\texttt{\{leonardo.takuno@gmail.com\}}}
\maketitle

\begin{enumerate}
\item Considere o seguinte algoritmo, cujo argumento $n$ � uma pot�ncia de 2. O algoritmo n�o faz nada �til.

\begin{codebox}
\Procname{$\proc{Algo}(n)$}
\li \If $n = 1$
\li \Then \Return 1
\End
\li \For $j \gets 1$ \To $8$
\li \Do $\id{z} \gets \proc{Algo}(n/2)$
\End
\li \For $i \gets 1$ \To $n^3$
\li \Do $\id{z} \gets 0$
\End
\end{codebox}

\begin{itemize}
\item[(a)] (0,5) Seja $T(n)$ o n�mero de vezes que a atribui��o $z \leftarrow 0$ � executada. Escreva uma recorr�ncia que define $T(n)$.
\item[(b)] (1,0) Mostre, \textbf{sem usar} o Teorema Mestre, que $T(n)$ � $\Omega(n^3 \log n)$.
\item[(c)] (1,0) Troque {\blue ``$8$''} por {\blue ``$7$''} no algoritmo e determine, usando o Teorema Mestre, a complexidade do algoritmo.
\end{itemize}

\item Um algoritmo age sobre que dependem de um par�metro $n$. Explique o significado das seguintes express�es:

\begin{itemize}
\item[(a)] (0,5) ``O consumo de tempo do algoritmo � $O(n^3 \log n)$''
\item[(b)] (0,5) ``O consumo de tempo do algoritmo � $\Omega(n^2)$''
\end{itemize}

\item 
\begin{itemize}
\item[(a)](1,5) Escreva um procedimento que recebe um vetor A de $n$ inteiros distintos e devolve o vetor A ordenado, em ordem crescente, usando \textit{Bubble Sort}.

\item [(b)](2,0) Prove a corretude do procedimento.

\item [(c)](2,0) Determine a complexidade de pior e de melhor caso. Indique para que tipo de inst�ncia o melhor e o pior caso ocorrem.  
\end{itemize}

\item Merge Sort

\begin{codebox}
\Procname{$\proc{Merge-Sort}(A, p, r)$}
\li \If $p < r$
\li \Then
        $q \gets \lfloor{(p + r) / 2} \rfloor$
\li $\proc{Merge-Sort}(A, p, q)$
\li $\proc{Merge-Sort}(A, q+1, r)$
\li $\proc{Merge}(A, p, q, r)$
\End
\end{codebox}

\item Inser��o em �rvore:

\begin{codebox}
\Procname{$\proc{Tree-Insert}(T,z)$}
\li $y \gets \const{nil}$
\li $x \gets \id{root}[T]$
\li \While $x \neq \const{nil}$
\li \Do
$y \gets x$
\li \If $\id{key}[z] < \id{key}[x]$
\li \Then $x \gets \id{left}[x]$
\li \Else $x \gets \id{right}[x]$
\End
\End
\li $p[z] \gets y$
\li \If $y = \const{nil}$
\li \Then
$\id{root}[T] \gets z$\>\>\>\>\>\>\>\>\Comment Tree $T$ was empty
\li \Else
\If $\id{key}[z] <\ id{key}[y]$
\li \Then $\id{left}[y]\ gets z$
\li \Else $\id{right}[y] \gets z$
\End
\End
\end{codebox}

\item outro:

\begin{codebox}
\Procname{$\proc{Segments-Intersect}(p_1, p_2, p_3, p_4)$}
\li $d_1 \gets \proc{Direction}(p_3, p_4, p_1)$
\li $d_2 \gets \proc{Direction}(p_3, p_4, p_2)$
\li $d_3 \gets \proc{Direction}(p_1, p_2, p_3)$
\li $d_4 \gets \proc{Direction}(p_1, p_2, p_4)$
\li \If $((d_1 > 0 \mbox{ and } d_2 < 0) \mbox{ or }
(d_1 < 0 \mbox{ and } d_2 > 0))$ and
\Indentmore
\zi $((d_3 > 0 \mbox{ and } d_4 < 0) \mbox{ or }
(d_3 < 0 \mbox{ and } d_4 > 0))$
\End
\li \Then \Return \const{\blue true}
\li \ElseIf $d_1 = 0$ {\blue and} $\proc{On-Segment}(p_3, p_4, p_1)$
\li \Then \Return \const{\blue true}
\li \ElseIf $d_2 = 0$ {\blue and} $\proc{On-Segment}(p_3, p_4, p_2)$
\li \Then \Return \const{\blue true}
\li \ElseIf $d_3 = 0$ {\blue and} $\proc{On-Segment}(p_1, p_2, p_3)$
\li \Then \Return \const{\blue true}
\li \ElseIf $d_4 = 0$ {\blue and} $\proc{On-Segment}(p_1, p_2, p_4)$
\li \Then \Return \const{\blue true}
\li \ElseNoIf \Return \const{\blue false}
\End
\end{codebox}

\end{enumerate}
\end{document}
