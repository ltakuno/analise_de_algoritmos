\documentclass{article}

\usepackage[brazil]{babel}     % Defini��o-padr�o l�ngua portuguesa
\usepackage[latin1]{inputenc}  % Para traduzir acentos
\usepackage{graphicx}
\usepackage{indentfirst}
\usepackage{float}

\usepackage{clrscode}

\usepackage{amsmath,amssymb,amsthm} \usepackage{amsfonts}
\usepackage[colorlinks,backref]{hyperref}

\setlength{\topmargin}{-0.5in}
\setlength{\textheight}{9in}
\setlength{\oddsidemargin}{-.125in}
\setlength{\evensidemargin}{-.125in}
\setlength{\textwidth}{6.75in}


\newcommand{\CWEB}{\texttt{CWEB} }
\newtheorem{definicao}{Defini��o}[section]
\newtheorem{exemplo}{Exemplo}[section]
\newtheorem{teorema}{Teorema}[section]
\newtheorem{corolario}{Corol�rio}[section]

\def\red{\color[rgb]{0.7,0,0}}
\def\green{\color[rgb]{0,.8,0}}
\def\darkgreen{\color[rgb]{0.1,0.4,0.0}}
\let\dgreen\darkgreen
\def\blue{\color[rgb]{0,0,.8}}
\def\yellow{\color[rgb]{1,1,0}}
\def\black{\color[rgb]{0,0,0}}

\def\pink{\color[rgb]{1,0,1}}
\def\brown{\color[rgb]{.5,.1,.3}}
\def\lilaz{\color[rgb]{.5,0,.5}}
\def\hmmm{\color[rgb]{.3,.1,.5}}
\def\magenta{\color[rgb]{.6,.05,.05}}

\newcommand{\uc}[1]{\centerline{\underline{#1}}}
\newcommand{\pic}[1]{\fbox{picture:{#1}}}
%\renewcommand{\bf}{\mbox{}}
\newcommand{\cP}{{\cal P}} \newcommand{\cT}{{\cal T}}
\newcommand{\add}{\mbox{\rm add}} \newcommand{\pr}{\mbox{\rm Pr}}

\def\stitle#1{\slidetitle{\red #1}\vspace{-0pt}}
\everymath={\blue}
\everydisplay={\blue}

\def\itemtrig{$\vartriangleright$}
\def\itemcirc{$\circ$}
\def\itemT{\item[\itemtrig]}
\def\itemC{\item[$\circ$]}

\renewcommand{\For}{\textbf{\blue for} }
\renewcommand{\To}{\textbf{\blue to} }
\renewcommand{\Do}{\>\>\textbf{\blue do}\hspace*{-0.7em}\'\addtocounter{indent}{1}}
\renewcommand{\While}{\textbf{\blue while} }
\renewcommand{\If}{\textbf{\blue if} }
\renewcommand{\Then}{\>\textbf{\blue then}\>\addtocounter{indent}{1}}
\renewcommand{\Else}{\kill\addtocounter{indent}{-1}\liprint\>\textbf{\blue else}\>\addtocounter{indent}{1}}
\renewcommand{\ElseIf}{\kill\addtocounter{indent}{-1}\liprint\textbf{\blue elseif} }
\renewcommand{\ElseNoIf}{\kill\addtocounter{indent}{-1}\liprint\textbf{\blue else} \addtocounter{indent}{1}}
\renewcommand{\Return}{\textbf{\blue return} }



\begin{document}


\bibliographystyle{plain}       % estilo de bibliografia

\title{Centro Universit�rio Senac\\ Bacharelado em Ci�ncia da Computa��o \\ An�lise e projeto de algoritmos}
\author{Professor: Leonardo Takuno \\
\texttt{\{leonardo.takuno@gmail.com\}}}
\maketitle

\begin{enumerate}
\item Prove a corretude do algoritmo de Euclides, que determina o M�ximo Divisor Comum (MDC) entre dois n�meros naturais. 
{\blue
\begin{verbatim}
mdc(m,n)
1  x = m
2  y = n
3  enquanto y != 0 fa�a
4    r = x mod y
5    x = y
6    y = r
7  devolva x
\end{verbatim}
}

\item Demonstre a corretude do algoritmo a seguir, que converte um n�mero inteiro na base decimal para a sua representa��o em bin�rio, ou seja, com 0's e 1's.

{\blue
\begin{verbatim}
converte(n)
1  t = n
2  k = -1
3  enquanto t > 0 fa�a
4    k = k + 1
5    b[k] = t mod 2
6    t = t div 2
7  devolva b
\end{verbatim}
}

\item Descreva (em pseudoc�digo) a vers�o recursiva da t�cnica de busca bin�ria para um \textit{array} ordenado e mostre a corretude deste algoritmo.

\item Use o m�todo de �rvore de recurs�o para determinar o tempo de execu��o dos algoritmos expressos pelas recorr�ncias abaixo:

\begin{enumerate}
\item $T(n) = 3 T(\lfloor n/2 \rfloor) + \Theta(n)$
\item $T(n) = 4 T(\lfloor n/2 \rfloor) + \Theta(n)$
\item $T(n) = 2 T(\lfloor n/2 \rfloor) + \Theta(n^2)$
\item $T(n) = 2 T(\lfloor n/2 \rfloor) + \Theta(n)$ (Mergesort)
\item $T(n) = 2 T(\lfloor n/3 \rfloor) + \Theta(n)$ 
\end{enumerate}

\end{enumerate}
\end{document}
